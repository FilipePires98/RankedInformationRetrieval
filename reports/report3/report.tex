
\documentclass[12pt]{article}

\usepackage{scicite}
\usepackage{times}
\usepackage{graphicx}
\usepackage{hyperref}
\usepackage{enumitem}
\usepackage{amsmath}

\topmargin -1.5cm
\oddsidemargin 0.0cm
\textwidth 16cm 
\textheight 23.5cm
\footskip 1.0cm

\newenvironment{sciabstract}{%
\begin{quote} \bf}
{\end{quote}} 

\newcounter{lastnote}
\newenvironment{scilastnote}{%
  \setcounter{lastnote}{\value{enumiv}}%
  \addtocounter{lastnote}{+1}%
  \begin{list}%
  {\arabic{lastnote}.}
  {\setlength{\leftmargin}{.22in}}
  {\setlength{\labelsep}{.5em}}
}
{\end{list}}

\title{Assignment 3} 

\author
{Filipe Pires [85122], João Alegria [85048]\\
\\
Information Retrieval\\
\normalsize{Department of Electronics, Telecommunications and Informatics}\\
\normalsize{University of Aveiro}\\
} 

\date{\today{}}

%%%%%%%%%%%%%%%%% END OF PREAMBLE %%%%%%%%%%%%%%%%

\begin{document} 
\baselineskip18pt
\maketitle 

\section*{Introduction}

This report was written for the discipline of 'Information Retrieval' and 
describes the implementation and evaluation of a ranked retrieval method that 
uses the indexes created with the solutions developed for the previous assignments.

We include the correction of design flaws of the delivery done prior to this one
and the updates applied both to the text corpus indexation and to our class diagram.
We also provide the instructions on how to run our code.

Along with the description of the solution, we also present the results of our
calculations to evaluate the solution and determine its efficiency according to 
the metrics proposed for this last assignment \cite{assign3}.
All code and documentation is present in our public GitHub project at 
\url{https://github.com/joao-alegria/RI}. 

\newpage
\section*{1. Re-Indexing the Corpus}

The first step to index the document corpus with the format intented was to 
consider not only the title of each document but also the abstract

\section*{2. Ranked Retrieval of Relevant Documents}

With an entirely functional index creator and a folder of generated indexes from
the corpus of text documents, it was now time to develop a program capable of
interpreting queries and returning the index entries most relevant to them.
In this chapter we describe our implementation of a query results ranked retriever
- that we called \texttt{Searcher} -, explain how we prepared it for memory 
limitations and present the updates done to our class diagram.

The file \texttt{Searcher.py} contains an abstract class called \texttt{Searcher}
that serves as a template for the implementations of results retriever classes.
For our purposes, we developed \texttt{IndexSearcher}, a class that extends from
the abstract template and is capable of selecting which index files will be 
required to answer a given query, assigning scores to documents from the index
and returning the documents considered most relevant to the query according to 
these scores.



\section*{3. Evaluation and Results Discussion}

....................

\section*{Conclusions}

After completing the assignment, we drew a few conclusions regarding our
solutions and the whole concept of ...........

The biggest challenge we faced was .........

From this assignment, we take ........

The overall perspective of our performance regarding the project is ..........

\begin{thebibliography}{9}
  \bibliographystyle{Science}

  \bibitem{assign3}
    S. Matos,
    \textit{IR: Assignment 3},
    University of Aveiro,
    2019/20.
  
\end{thebibliography}

\clearpage

\end{document}




















